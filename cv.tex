%------------------------
% Resume Template
% Author : Anubhav Singh
% Github : https://github.com/xprilion
% License : MIT
%------------------------

\documentclass[a4paper,20pt]{article}

\usepackage{latexsym}
\usepackage[empty]{fullpage}
\usepackage{titlesec}
\usepackage{marvosym}
\usepackage[usenames,dvipsnames]{color}
\usepackage{verbatim}
\usepackage{enumitem}
\usepackage[pdftex]{hyperref}
\usepackage{fancyhdr}

\pagestyle{fancy}
\fancyhf{} % clear all header and footer fields
\fancyfoot{}
\renewcommand{\headrulewidth}{0pt}
\renewcommand{\footrulewidth}{0pt}

% Adjust margins
\addtolength{\oddsidemargin}{-0.530in}
\addtolength{\evensidemargin}{-0.375in}
\addtolength{\textwidth}{1in}
\addtolength{\topmargin}{-.45in}
\addtolength{\textheight}{1in}

\urlstyle{rm}

\raggedbottom
\raggedright
\setlength{\tabcolsep}{0in}

% Sections formatting
\titleformat{\section}{
  \vspace{-10pt}\scshape\raggedright\large
}{}{0em}{}[\color{black}\titlerule \vspace{-6pt}]

%-------------------------
% Custom commands
\newcommand{\resumeItem}[2]{
  \item\small{
    \textbf{#1}{: #2 \vspace{-2pt}}
  }
}

\newcommand{\resumeItemWithoutTitle}[1]{
  \item\small{
    {#1 \vspace{-2pt}}
  }
}

\newcommand{\resumeSubheading}[4]{
  \vspace{-1pt}\item
    \begin{tabular*}{0.97\textwidth}{l@{\extracolsep{\fill}}r}
      \textbf{#1} & #2 \\
      \textit{#3} & \textit{#4} \\
    \end{tabular*}\vspace{-5pt}
}


\newcommand{\resumeSubItem}[2]{\resumeItem{#1}{#2}\vspace{-3pt}}

\renewcommand{\labelitemii}{$\circ$}

\newcommand{\resumeSubHeadingListStart}{\begin{itemize}[leftmargin=*]}
\newcommand{\resumeSubHeadingListEnd}{\end{itemize}}
\newcommand{\resumeItemListStart}{\begin{itemize}}
\newcommand{\resumeItemListEnd}{\end{itemize}\vspace{-5pt}}

%-----------------------------
%%%%%%  CV STARTS HERE  %%%%%%

\begin{document}

%----------HEADING-----------------
\begin{tabular*}{\textwidth}{l@{\extracolsep{\fill}}r}
  \textbf{{\LARGE Simon Efremov}} & Email: \href{mailto:esemiko@gmail.com}{esemiko@gmail.com}\\
  \href{https://github.com/esemi}{Github: ~~github.com/esemi} & Mobile:~~~+420-604-872-966 \\
  \href{https://www.linkedin.com/in/esemi/}{Linkedin: linkedin.com/in/esemi} & Prague, Czech Republic \\


\end{tabular*}

% TODO не слишком ли перекос в сторону бизнеса и импакта? Как то без техники совсем грустно =(
% TODO нужно ли добавить описание компаний
% TODO шаблон сильно лучше? https://docs.google.com/document/d/1-7QVHhYfiMZmTe5L1P6PEs10NZnXSCGHcyN9P72APHk/edit
% TODO не нужно ли поменять местами название компании с тайтлом воркера?
% TODO не пестрит ли в глазах от форматирования на всей странице?
%-----------EXPERIENCE-----------------
\vspace{-5pt}
\section{Experience}
  \resumeSubHeadingListStart
    \resumeSubheading{Gyre TV}{Remote}
    {Chief Technology Officer}{Fall 2021 - Present}
    \resumeItemListStart
        \resumeItem{Rebuild the data collection system}
        {Supported a 50x growth in user base and streams.}
        \resumeItem{Moved part of the infrastructure to another hosting provider}
        {Reduced servers expenses by 44\%.}
        \resumeItem{Tuned the development process}
        {Introduced a unified kanban board with formalised user-stories.}
        \resumeItem{Released the first version of the service}
        {From POC to production-ready service with live customers.}
    \resumeItemListEnd

    \vspace{5pt}

    \resumeSubheading{Semrush Inc.}{Prague, Czech Republic}
    {Software Developer at collaboration project}{Winter 2022 - Present}
    \resumeItemListStart
        \resumeItem{Improved the on-boarding process for new teammates}
        {Updated project documentation, consolidated coding standards through linters and automated all manual activities through CI/CD.}
        \resumeItem{Standardised tasks for integration with other teams}
        {Accelerated new integration's into our service.}
        \resumeItem{Implemented a new access distribution model for corporate clients}
        {Made the user experience in our tool much simpler for 30\% of users.}
        \resumeItem{Did extensive research on Zapier-like services}
        {Became clear how current integrations with 3rd-party tools could be built on.}
    \resumeItemListEnd

    \vspace{5pt}

    \resumeSubheading{Mobile Apps Studio}{Remote}
    {Software Developer at wellness mobile app}{Summer 2022}
    \resumeItemListStart
        \resumeItem{Developed and implemented backend wellness applications for iOS and Android}
        {Engaged the first 500 users in closed alpha testing.}
        \resumeItem{Implemented natural language processing of user requests}
        {Transcription quality increased by 37\% (as rated by customers).}
    \resumeItemListEnd

    \vspace{5pt}

    \resumeSubheading{Semrush Inc.}{Prague, Czech Republic}
    {Tech lead at billing platform}{Summer 2020 - Winter 2022}
    \resumeItemListStart
        \resumeItem{Adopted a change in the sales model for our SaaS}
        {Designed and implemented the POC of the new subscription system.}
        \resumeItem{Migrated the team's services to the cloud infrastructure}
        {Reduced infrastructure expenses by 60\%.}
        \resumeItem{Supported the expansion of the team}
        {Recruited and integrated 5 colleagues, which increased the productivity of the team exponentially.}
    \resumeItemListEnd

    \vspace{5pt}

    \resumeSubheading{Different companies}{St. Petersburg, Russia}
    {Software Developer}{Summer 2008 - Summer 2020}
    \resumeItemListStart
        \resumeItemWithoutTitle
        {Previous experiences are accessible by request.}
    \resumeItemListEnd
\resumeSubHeadingListEnd
\vspace{-5pt}


%-----------EDUCATION-----------------
% стоит ли добавить тему диплома? Не принято ли так на западе?
\section{Education}
    \resumeSubHeadingListStart
        \resumeSubheading
        {University Information Technologies, Mechanic and Optics (University ITMO)}{St. Petersburg, Russia}
        {Specialist Degree in Information Systems and Technology}{September 2006 - May 2012}
    \resumeSubHeadingListEnd
\vspace{-5pt}

%-----------PROJECTS-----------------
% TODO нужно ли в проекты впихивать рабочие проекты? Или лучше описать их в контексте описания места работы?
% TODO нужно ли добавить ссылки на статьи/сервисы, если они на русском (сервисы)?
% TODO на чём должен быть фокус в описании проекта - что сделал или зачем проект нужен и какую проблему решает?
\section{Projects}
    \resumeSubHeadingListStart

    %\resumeSubItem{IMDB Visualiser (Visualisation, Graph Theory)}{Interactive graph of films and actors, based on links between them on imdb.com. Tech: Golang, Neo4j, Javascript. (work in progress)}
    %\vspace{2pt}

    \resumeSubItem{Tax Calculator Tool (Investing, Taxes, CLI-tools)}{Utility for tax reporting based on Interactive Brokers LLC report for residents of the Czech Federation. Tech: Python, Pandas. (May '22)}
    \vspace{2pt}

    \resumeSubItem{Investment portfolio management tool (Investing, CLI-tools)}{Portfolio management software based on reports from five brokerages. Tech: Scala, Javascript, XPath. (April '21)}
    \vspace{2pt}

    \resumeSubItem{Crypto Arbitrage Bot (Cryptocurrency, Trading)}{The bots looked for exchange rate discrepancies between five cryptocurrency exchanges and placed orders for profit. Tech: Python, Redis, Aiohttp, lxml. (June '20)}
    \vspace{2pt}

    \resumeSubItem{Dating Search Face (Crawlers, Search Engine, Face Detection, Face Recognition)}{A system to find a person by facial image from tinder.com users. Tech: Python, Redis, AWS, OpenCV, Scikit, FFmpeg, Dlib. (March '19)}
    \vspace{2pt}

    \resumeSubItem{Adult Movies Graph (Crawlers, Visualisation, Graph Theory, Web Services)}{An interactive graph of pornhub.com clips based on backlinks. Tech: Python, Javascript, lxml, Puppeteer, Mongodb, Sigma.js, Gephi. (July '18)}
    \vspace{2pt}

    \resumeSubItem{Musicians Clustering (Crawlers, Music, Research)}{Research into the overlap of music genres on the Yandex.Music (Spotify-like service). Tech: Python, Selenium, lxml, Igraph (November '17)}
    \vspace{2pt}

    \resumeSubItem{Travian Bot (Browser Automation, Gaming)}{Bot to automate most of the daily tasks in Travian online-game. Tech: Python, Selenium, lxml (February '17)}
    \vspace{2pt}

    \resumeSubItem{DSeye - An analytics system for online-game (Crawlers, Visualisation, Web Services, Gaming)}{Collect and display metrics for all players on the MMORTS Destiny Sphere. Tech: PHP, Python, Mysql. (January '14)}

\resumeSubHeadingListEnd
\vspace{-5pt}

%-----------SKILLS-----------------
% TODO список базвордов бесит и хочется его уменьшить, оставил только те, с которыми есть либо обширный опыт либо желание работатьс. Это норм вообще или нужен индусский подход для автопарсеров?
% TODO Непонятно, как запруфать каждый базворд в описании проектов/мест работы и нужно ли это
\section{Skills Summary}
	\resumeSubHeadingListStart
	\resumeSubItem{Languages}{~~~~~~~Python, SQL, PHP}
	\resumeSubItem{Tools}{~~~~~~~~~~~~~~FastAPI, Starlette, Flask, SQLAlchemy, Peewee, Pytest, Poetry}
	\resumeSubItem{Data}{~~~~~~~~~~~~~~~Postgresql, Redis, Mysql, Mongodb, Neo4j}
	\resumeSubItem{Infrastructure}{~Kubernetes, Helm, Docker, GCP, Gitlab CI, Github Actions}
\resumeSubHeadingListEnd

\end{document}